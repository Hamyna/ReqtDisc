% !TEX root = knauss-vissuelizer.tex
\section{V:Issue:lizer}


\viss\ is an interactive tool that allows software developers and managers to dynamically explore the discussion of requirements  in online repositories with a focus on highlighting the difference between clarification and implementation related communication.


The main window shows a list of requirements on the  left (e.g. workitems in jazz, items in jira, issues in other systems) (see Figure \ref{fig:screenshot}).
\viss\ adds visualizations to the selected discussions in the centre or in an extra window. 
These visualizations help to assess the communication through discussion events (e.g. comments) related to these requirements.
The panel on the right allows users to adjust parameters of the visualization.
Most importantly, the resolution of time intervals for the visualization can be adjusted, either as a fixed number (here: three and eight intervals), or as a fixed time interval (e.g. days, weeks, month).
%
\viss\ currently supports two different visualizations: 

\subsubsection{Clarification Trajectories} 
This visualization shows how the percentage of clarification events to other discussion events related to a requirement changes over its lifetime.
As the visualization of the clarification trajectory (c.f. Figure \ref{fig:example-trajectory}) is a new concept, it needs some explanation.
The black line represents the lifetime of the requirement discussion from the creation of the requirement in the system to the last recorded discussion event.
Dashed lines divide the lifeline into quarters and help to see in which part of the lifetime discussion occurs.
Discussion events are depicted by rectangles.
They are shown below the lifeline, if they are clarification events and above the lifeline if not.
A grey line shows the sum of clarification.
In a classic trajectory with clarification up-front and only implementation related communication in the end, this grey line will start in the bottom left and raise to the upper right corner.
% For increased readability, the different types of discussion events are coloured in the tool (clarification events: red, other: blue).

\subsubsection{Social Networks} 
This visualization shows the communication pattern of those participating in a requirement's discussion. The developers are presented as nodes; the connections between nodes are weighted by the amount of communication both developers share about the requirement in a specific time interval in a requirement's lifetime. The interval can be a ....  
\viss\ also displays the developer node as a pie chart to show the percentage of clarification vs. implementation-related communication in which the developer is involved.




% integrates information of the automatic analysis of online communication into the social networks, i.e. showing a pie chart with the percentage of clarification and implementation related communication for this developer.

